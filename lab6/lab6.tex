\documentclass[12pt]{article}

\usepackage{fullpage}
\usepackage{multicol,multirow}
\usepackage{tabularx}
\usepackage{ulem}
\usepackage[utf8]{inputenc}
\usepackage{mathtools}
\usepackage{pgfplots}
\usepackage[russian]{babel}

\begin{document}

\section*{Лабораторная работа №\,6 по курсу дискрeтного анализа: Калькулятор}

Выполнил студент группы М8О-308Б-22 МАИ \textit{Немкова Анастасия}.

\subsection*{Условие}
Необходимо разработать программную библиотеку на языке C или C++, реализующую простейшие арифметические действия и проверку условий над целыми неотрицательными числами. На основании этой библиотеки, нужно составить программу, выполняющую вычисления над парами десятичных чисел и выводящую результат на стандартный файл вывода.

Список арифметических операций:
\begin{itemize}
    \item Сложение (+).
    \item Вычитание (-).
    \item Умножение (*).
    \item Возведение в степень ($\wedge$).
    \item Равно (=).
\end{itemize}

Замечание: при реализации деления можно ограничить делитель цифрой внтуреннего представления «длинных» чисел, в этом случае максимальная оценка, которую можно получить за лабораторную работу, будет ограничена оценкой 3 («удовлетворительно»).

В случае возникновения переполнения в результате вычислений, попытки вычесть из меньшего числа большее, деления на ноль или возведении нуля в нулевую степень, программа должна вывести на экран строку Error.

Список условий:
\begin{itemize}
    \item Больше (>).
    \item Меньше (<).
    \item Равно (=).
\end{itemize}

В случае выполнения условия, программа должна вывести на экран строку true, в противном случае – false.

\subsection*{Метод решения}

Длинные числа храним в виде вектора его цифр в порядке убывания разрядов.

Сложение и вычитание выполняются в столбик. При сложении двух чисел каждый блок складывается отдельно, начиная с младших разрядов. Если сумма блока превышает 10000, сохраняется "перенос" в следующий разряд. В конце, если остался перенос, он добавляется как новый блок в результат. При вычитании проверяем, что уменьшаемое больше вычитаемого и также выполняем его по блокам, начиная с младших разрядов. Если текущий блок меньше, чем блок вычитаемого числа, необходимо "заимствовать" из следующего разряда, увеличивая его на 10000. Заимствование учитывается при вычислении разности текущего блока. Асимптотика обеих операций - O(n).

Для сравнения чисел сравниваем их длины, если они одинаковы, то сравниваем разряды. Эти операции также выполняются за O(n).

Умножение выполняется по аналогии с обычным умножением в столбик. Каждый блок первого числа умножается на каждый блок второго числа, и результаты складываются, учитывая позицию разряда. Если результат превышает 10000, сохраняется перенос для добавления в более старшие разряды. Умножение выполняется за O(n * m).

Алгоритм деления использует бинарный поиск для нахождения частного. От текущего остатка (в котором постепенно добавляются блоки делимого) ищется максимальный множитель делителя, который помещается в текущий блок результата. Остаток обновляется путем вычитания произведения делителя на найденный множитель. Сложность операции - O(n * m * log(a)), а - основание системы счисления.

Возведение в степень реализуется с использованием метода быстрого возведения в степень. Если показатель степени четный, число возводится в квадрат, и степень делится на два. Если степень нечетная, результат умножается на основание, а степень уменьшается на единицу. Сложность операции - O(nlog(m)).

\subsection*{Описание программы}

Для работы с длинными целыми числами был создан класс TBigInt. Он поддерживает основные арифметические операции, такие как сложение, вычитание, умножение, деление и возведение в степень. Также реализованы операторы сравнения и ввод-вывод через стандартные потоки.
\begin{itemize}
    \item TBigInt(const std::string& str) - конструктор числа из строки
    \item void DeleteZeros() - удаление лидирующих нулей
    \item friend TBigInt operator+ (const TBigInt& num1, const TBigInt& num2) - перегрузка опрератора сложения
    \item friend TBigInt operator- (const TBigInt& num1, const TBigInt& num2) - перегрузка опреатора вычитания
    \item friend TBigInt operator* (const TBigInt& num1, const TBigInt& num2) - перегрузка оператора умножения
    \item friend TBigInt operator/ (const TBigInt& num1, const TBigInt& num2) - перегрузка оператора деления
    \item friend TBigInt operator$\wedge$ (TBigInt num1, TBigInt num2) - перегрузка оператора возведения в степень
    \item friend bool operator== (const TBigInt& num1, const TBigInt& num2) - перегрузка оператора равно
    \item friend bool operator> (const TBigInt& num1, const TBigInt& num2) - перегрузка оператора больше
    \item friend bool operator< (const TBigInt& num1, const TBigInt& num2) - перегрузка оператора меньше
    \item friend std::istream& operator>> (std::istream& in,TBigInt& num) - перегрузка оператора ввода
    \item friend std::ostream& operator<< (std::ostream& out, const TBigInt& num) - перегрузка оператора вывода
\end{itemize}


\subsection*{Дневник отладки}

\begin{enumerate}
    \item 2 окт 2024, 20:09:54 WA на 3 тесте
		
    Неверно реализовано сложение. Решение: если после сложения всех разрядов остался непустой перенос, он добавляется как новый старший разряд числа


\end{enumerate}

\subsection*{Тест производительности}

Для измерения производительсти протестируем выполнение арифметических операций: сложения, вычитания и умножения на входных данных различной длины. Возьмем длины чисел : 10, 50, 100, 500, 1000 и будем замерять группами по 100 операций.
	
\begin{tikzpicture}% coordinates
    \begin{axis}[
        xlabel=Длина числа,
        ylabel=время выполнения операции (ms),
        ymin = 0,
        grid=major,
        legend pos=outer north east
        ]
        \legend{Sum, Sub, Mult, Div}
        \addplot coordinates {(10, 29) (50, 32) (100, 54) (500, 135) (1000, 267)};
        \addplot coordinates {(10, 28) (50, 46) (100, 68) (500, 222) (1000, 440)};
        \addplot coordinates {(10, 0) (50, 1) (100, 18) (500, 300) (1000, 1057)};
        \addplot coordinates {(10, 3) (50, 24) (100, 62) (500, 875) (1000, 3947)};
    \end{axis}
\end{tikzpicture}


\subsection*{Выводы}

В ходе лабораторной работы была реализована библиотека для работы с большими целыми числами, которая поддерживает такие основные арифметические операции, как сложение, вычитание, умножение, деление и возведение в степень. Реализации данных операций не единственны, существуют альтернативные алгоритмы, такие как умножение Карацубы, которое ускоряет процесс умножения за счет рекурсивного деления числа на части. Существуют также более продвинутые алгоритмы для деления, такие как метод Ньютона, который может ускорить вычисления за счет более эффективного поиска частного.

\end{document}



